% \documentclass[a4paper]{jarticle} % 一般的なスタイルの書き方
\documentclass[a4paper]{ujarticle} % 2段構成のスタイル
%\documentclass[a4paper]{jreport} %卒論原稿はこのスタイル
\setlength{\topmargin}{-2.04cm}%例:上余白を設定
\setlength{\oddsidemargin}{-1.04cm}%例:左余白を1.5cmにする
\setlength{\evensidemargin}{-1.04cm}%例b:左余白を1.5cmにする
\setlength{\textwidth}{18cm}%例:一行の幅を18cmにする
\setlength{\textheight}{25cm}%例:一ページの文章の縦の長さを25cmにする
%\setlength{\textwidth}{45em}%例:一行の文字数を45文字にする(未使用)

%%%%%%%%%%%%%%%%%%%%%%%%%%
%% usepaclagae 群
%%%%%%%%%%%%%%%%%%%%%%%%%%
\usepackage{amsmath,bm} %多次元空間ベクトルRを表記するのに必要
\usepackage{amsfonts}
\usepackage{ascmac} %枠付き文章を表記するのに必
\usepackage{amssymb}
% \mathbb{R}^{l} %表記例
\usepackage{algorithm}
% \usepackage{algorithmicx}
\usepackage{algpseudocode}
\usepackage[dvipdfmx]{graphicx}
\usepackage[dvipdfmx]{color}
\usepackage{here} %[hbtp]の代わりに[H]と書きこむと強制的にその場所に図や表を挿入す
\pagestyle{empty}%ページ番号を表示しない

%%%%%%%%%%%%%%%%%%%%%%%%%
\newcommand{\argmax}{\mathop{\rm arg~max}\limits}
\newcommand{\argmin}{\mathop{\rm arg~min}\limits}
\newcommand{\bX}{\bm{X}}
\newcommand{\bmu}{\bm{\mu}}
\newcommand{\bSigma}{\bm{\Sigma}}
\newcommand{\bx}{\bm{x}}
\newcommand{\by}{\bm{y}}
\newcommand{\zl}{\rightarrow}
\newcommand{\zh}{\leftarrow}

%%%%%%%%%%%%%%%%%%%%%%%%%

\newtheorem{lem}{補題}

\makeatletter
\def\@maketitle{%
\begin{center}%
{\LARGE \@title \par}% タイトル
\end{center}%
\begin{flushright}%
{\large \@date}% 日付
\end{flushright}%
\begin{flushright}%%
{\large \@author}% 著者
\end{flushright}%
\par\vskip 1.5em
}
\makeatother
\title{Meanshift Segmentation} %ここにタイトルを記入すること.
\date{\today}
\author{大森 夢拓}

\begin{document}
	入力
	\begin{itemize}
		\item $X \in \mathbb{R}^{nd}$: 画像データベクトル(浮動小数)
	\end{itemize}
	出力
	\begin{itemize}
		\item $Y \in \mathbb{R}^{nd}$: アルゴリズム適用後の画像データベクトル(浮動小数)
	\end{itemize}
	ハイパーパラメータ
	\begin{itemize}
		\item $h_s$: 画像空間の周辺範囲
		\item $h_r$: 輝度値の周辺範囲
		\item $N$: meanshiftのループ回数
	\end{itemize}
	その他
	\begin{itemize}
		\item $x$: ピクセルのオブジェクト. 以下のデータ構造を持つ.
		\begin{itemize}
			\item value: 輝度値
			\item index: $X$ 上での $x$ の位置(添字)
		\end{itemize}
	\end{itemize}

	\begin{algorithm}[H]
		\caption{Meanshift Segmentation}
		\label{alg:mss}
		\begin{algorithmic}[1]
			\Function{meanshift segmentation}{$X$}
				\State{$Y = $ \Call{segmentation}{X}}
			\EndFunction
		\end{algorithmic}
	\end{algorithm}
	
	\begin{algorithm}[H]
		\caption{Segmentation}
		\label{alg:s}
		\begin{algorithmic}[1]
			\Function{segmentation}{X}
				\For{$x_i \in$ X}
					\State{$Y$[$x_i$.index].value $=$ \Call{meanshift}{$x_i$}} \Comment{$x_i$ の輝度値を周辺局所範囲内のピーク値へ更新し, $Y$ へ代入}
				\EndFor
				\State{\Return{Y}}
			\EndFunction
		\end{algorithmic}
	\end{algorithm}

	\begin{algorithm}[H]
		\caption{Meanshift}
		\label{alg:ms}
		\begin{algorithmic}[1]
			\Function{meanshift}{$v_i$} \Comment{輝度値 $v_i$ を 更新するアルゴリズム}
				\For{$n = 0, \ldots , N$} \Comment{$v_i$ を中心とする周辺局所範囲に基づいた更新を $N$ 回行う}
					\State{$v_{sum} = 0$, $S =$ []} \Comment{周辺局所範囲のピクセルの集合 $S$ と $S$ 内の輝度値の合計の初期化}
					\State{$S_{h_s}$ $=$ \Call{make Shs}{$x_i, X$}} \Comment{$h_s$ に基づいた局所範囲内のピクセルの集合を作成}
					\State{$S_{h_s, h_r}$ $=$ \Call{make Shr}{$x_i, S_{h_s}$}} \Comment{$h_r$ に基づいた局所範囲内のピクセルの集合を作成}
					
					\If{$|S_{h_s, h_r}|$ == 0}
						\State{break}
					\EndIf
					\State{$x.\mathrm{value} =$ \Call{mean value}{$S_{h_s, h_r}$}} \Comment{$x$.value を周辺範囲内のピクセルの輝度値の平均へ更新}
				\EndFor
				\State{\Return{$v_i$}}
			\EndFunction
		\end{algorithmic}
	\end{algorithm}
			
	\begin{algorithm}[H]
		\caption{make Shs}
		\label{alg:mss}
		\begin{algorithmic}[1]
			\Function{make Shs}{$x, X$}
				\For{$x_i \in$ X} \Comment {$X$ 内の全てのピクセルを対象に $x$ を中心とする画像空間周辺範囲の内外判定}
					\State{dif $= |x.\mathrm{value} - x_s.\mathrm{value}|$} \Comment{周辺範囲の中心の輝度値 $x$.value と輝度値 $x_s$.value の差}
					\If{dif $\le h_r$}
						\State{$S \leftarrow x_s$} \Comment{$x$ が周辺範囲内のピクセルであれば集合 $S$ へ追加}
					\EndIf
				\EndFor
			\EndFunction
		\end{algorithmic}
	\end{algorithm}

	\begin{algorithm}[H]
		\caption{make Shr}
		\label{alg:mss}
		\begin{algorithmic}[1]
			\Function{make Shr}{$x, S$}
				\For{$x_s \in$ S} \Comment {$S$ 内の全ての輝度値を対象に $v_i$ を中心とする色空間周辺範囲の内外判定}
					\State{dif $= |x.\mathrm{value} - x_s.\mathrm{value}|$} \Comment{周辺範囲の中心の輝度値 $x$.value と輝度値 $x_s$.value の差}
					\If{dif $\le h_r$}
						\State{$S \leftarrow x_s$} \Comment{$x$ が周辺範囲内のピクセルであれば集合 $S$ へ追加}
					\EndIf
				\EndFor
			\EndFunction
		\end{algorithmic}
	\end{algorithm}

\end{document}
