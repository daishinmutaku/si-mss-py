% \documentclass[a4paper]{jarticle} % 一般的なスタイルの書き方
\documentclass[a4paper]{ujarticle} % 2段構成のスタイル
%\documentclass[a4paper]{jreport} %卒論原稿はこのスタイル
\setlength{\topmargin}{-2.04cm}%例:上余白を設定
\setlength{\oddsidemargin}{-1.04cm}%例:左余白を1.5cmにする
\setlength{\evensidemargin}{-1.04cm}%例b:左余白を1.5cmにする
\setlength{\textwidth}{18cm}%例:一行の幅を18cmにする
\setlength{\textheight}{25cm}%例:一ページの文章の縦の長さを25cmにする
%\setlength{\textwidth}{45em}%例:一行の文字数を45文字にする(未使用)

%%%%%%%%%%%%%%%%%%%%%%%%%%
%% usepaclagae 群
%%%%%%%%%%%%%%%%%%%%%%%%%%
\usepackage{amsmath,bm} %多次元空間ベクトルRを表記するのに必要
\usepackage{amsfonts}
\usepackage{ascmac} %枠付き文章を表記するのに必
\usepackage{amssymb}
% \mathbb{R}^{l} %表記例
\usepackage{algorithm}
% \usepackage{algorithmicx}
\usepackage{algpseudocode}
\usepackage[dvipdfmx]{graphicx}
\usepackage[dvipdfmx]{color}
\usepackage{here} %[hbtp]の代わりに[H]と書きこむと強制的にその場所に図や表を挿入す
\pagestyle{empty}%ページ番号を表示しない

%%%%%%%%%%%%%%%%%%%%%%%%%
\newcommand{\argmax}{\mathop{\rm arg~max}\limits}
\newcommand{\argmin}{\mathop{\rm arg~min}\limits}
\newcommand{\bX}{\bm{X}}
\newcommand{\bmu}{\bm{\mu}}
\newcommand{\bSigma}{\bm{\Sigma}}
\newcommand{\bx}{\bm{x}}
\newcommand{\by}{\bm{y}}
\newcommand{\zl}{\rightarrow}
\newcommand{\zh}{\leftarrow}

%%%%%%%%%%%%%%%%%%%%%%%%%

\newtheorem{lem}{補題}

\makeatletter
\def\@maketitle{%
\begin{center}%
{\LARGE \@title \par}% タイトル
\end{center}%
\begin{flushright}%
{\large \@date}% 日付
\end{flushright}%
\begin{flushright}%%
{\large \@author}% 著者
\end{flushright}%
\par\vskip 1.5em
}
\makeatother
\title{Meanshift Segmentation} %ここにタイトルを記入すること.
\date{\today}
\author{大森 夢拓}

\begin{document}
	入力
	\begin{itemize}
		\item $X \in \mathbb{R}^{col \times row}$: 画像データ行列(浮動小数)
	\end{itemize}
	出力
	\begin{itemize}
		\item $Y \in \mathbb{R}^{col \times row}$: アルゴリズム適用後の画像データ行列(浮動小数)
	\end{itemize}
	ハイパーパラメータ
	\begin{itemize}
		\item $h_s$: 画像空間の周辺範囲
		\item $h_r$: 輝度値の周辺範囲
		\item $N$: meanshiftのループ回数
	\end{itemize}

	\begin{algorithm}[H]
		\caption{Meanshift Segmentation}
		\label{alg:mss}
		\begin{algorithmic}[1]
			\Function{meanshift segmentation}{$X$}
				\State{$Y = $ \Call{segmentation}{X}}
			\EndFunction
		\end{algorithmic}
	\end{algorithm}
	
	\begin{algorithm}[H]
		\caption{Segmentation}
		\label{alg:s}
		\begin{algorithmic}[1]
			\Function{segmentation}{X}
				\For{$y = 0$, $y < |X.col|$, $y++$}
					\For{$x = 0$, $x < |X.row|$, $x++$}
						\State{$Y$[y][x].value $=$ \Call{meanshift}{$x$, $y$, $X$}} \Comment{$X$ 輝度値を周辺局所範囲内のピーク値へ更新し, $Y$ へ代入}
					\EndFor
				\EndFor
				\State{\Return{Y}}
			\EndFunction
		\end{algorithmic}
	\end{algorithm}

	\begin{algorithm}[H]
		\caption{Meanshift}
		\label{alg:ms}
		\begin{algorithmic}[1]
			\Function{meanshift}{$x$, $y$, $X$} \Comment{$v$ を更新するアルゴリズム}
				\State{$v = X[y][x]$}
				\For{$n = 0, \ldots , N$} \Comment{$x$ を中心とする周辺局所範囲に基づいた更新を $N$ 回行う}
					\State{$S$, $x$, $y$, $v$ $=$ \Call{make S}{$x, y, v, X$}} \Comment{周辺範囲内のピクセルの集合の作成と各中央値の更新}
					\If{$|S|$ == 0}
						\State{break}
					\EndIf
				\EndFor
				\State{\Return{$v$}}
			\EndFunction
		\end{algorithmic}
	\end{algorithm}

	\begin{algorithm}[H]
		\caption{make S}
		\label{alg:mss}
		\begin{algorithmic}[1]
			\Function{make Shs}{$x_c, y_c, v, X$}
			\State{$S = []$} \Comment {画像空間周辺範囲内のピクセルの集合 $S$ の初期化}
			\State{$x_{sum} = 0$}
			\State{$y_{sum} = 0$}
			\State{$y_{min} = \max(0, y_c - h_s)$} \Comment{周辺範囲での $y$ 軸のLower}
			\State{$y_{max} = \min(|X.\mathrm{col}| - 1, y_c + h_s)$} \Comment{周辺範囲での $y$ 軸のUpper}
			\State{$x_{min} = \max(0, x_c - h_s)$} \Comment{周辺範囲での $x$ 軸のLower}
			\State{$x_{max} = \min(|X.\mathrm{row}| - 1, x_c + h_s)$} \Comment{周辺範囲での $x$ 軸のUpper}
			\For{$y = y_{min}$, $y <= y_{max}$, $y++$}
				\For{$x = x_{min}$, $x <= x_{max}$, $x++$}
					\State{$d = |X[y][x] - v|$} \Comment{周辺範囲の中心の輝度値との差}
					\If{$d \le h_r$}
						\State{$y_{sum} += y$}
						\State{$x_{sum} += x$}
						\State{$S \leftarrow X[y][x]$} \Comment{周辺範囲内のピクセルであれば集合 $S$ へ追加}
					\EndIf
				\EndFor
			\EndFor
			\State{$x_{mean} = \frac{x_{sum}}{|S|}$, $y_{mean} = \frac{y_{sum}}{|S|}$, $v_{mean} = mean(S)$}
			\State{\Return{$S$, $x_{mean}$, $y_{mean}$, $v_{mean}$}}
			\EndFunction
		\end{algorithmic}
	\end{algorithm}

\end{document}
